\documentclass{article}
\usepackage[utf8]{inputenc}

\title{LABWORK6}
\author{Damien ALVAREZ DE TOLEDO }
\date{November 2019}

\begin{document}

\maketitle

\section{Implementation of the Labworks}


\textbf{Binarization :}\newline

Binarization was done with the \textbf{binarize} kernel function : a threashold is declared before executing the function.\newline
All of the pixels with a grey value higher than the threashold get to be white pixels. Otherwise they turn into black pixels.\newline
\newline

\textbf{Brightness :}\newline

Brightness is adjusted on a picture with the \textbf{brightness} kernel function : a brightness value is declared before executing the function. \newline
Pixels get their grey value increased if the brightness is higher than 50, by an amount of 100-brightness. Grey value gets decreased if brightness is lower than 50, by 50-brightness. Picture gets totally white if brightness = 100, pitch black if brightness = 0 and no modifications if brightness = 50.
\newline

\textbf{Blending :}\newline

Blending is performed on two picture with \textbf{blendingGray} kernel function : a coefficient allows to know what percentage of display will take each picture on the final output picture.\newline
\textbf{labwork.h} has been modified to fulfill this function : a new variable of image input has been added because two inputs are needed for this image processing. Other functions were subsequently added. \newline
\newline

\section{Experimenting with different 2D Block Size Values}

\textbf{Time of Execution Comparison :}\newline

\textbf{(32 x 32) :}\newline

Time of Execution (average of 5 executions) : 139,8 ms\newline 

\textbf{(16 x 16) :}\newline

Time of Execution (average of 5 executions) : 130 ms\newline

\textbf{Conclusion :}\newline

We can conclude that time of execution is slightly faster for 16x16 blocks.\newline





\end{document}

