\documentclass{article}
\usepackage[utf8]{inputenc}

\title{LABWORK: 1}
\author{Damien ALVAREZ DE TOLEDO }
\date{November 2019}

\begin{document}

\maketitle

\section{How to implement the conversion :}\newline

The following command lines from the labwork1\_CPU() function \newline 
are copied in the labwork1\_OpenMP() function : 
\newline

\textbf{  \$ for (int j = 0; j < 100; j++) \{  \newline  
  \$     for (int i = 0; i < pixelCount; i++) \{\newline
  \$         outputImage[i * 3] = (char) (((int) inputImage->buffer[i * 3] + (int) \newline	inputImage->buffer[i * 3 + 1] + (int) inputImage->buffer[i * 3 + 2]) /\newline 	3);\newline
          \$  outputImage[i * 3 + 1] = outputImage[i * 3];\newline
          \$ outputImage[i * 3 + 2] = outputImage[i * 3];\newline
      \$ \}\newline
  \$  \}\newline}
  \newline 
  
  with the added command above the copied fragment : \newline
  
  \textbf{\# pragma omp parallel for} \newline
  
  In the upward command, the \textbf{"parallel"} argument is used to parallelize the tasks of the labwork1\_OpenMP() algorithm and the \textbf{"for"} argument is used to optimize the parallelization in the "for" loops. \newline 

\section{Speedup results:}\newline

The speedup is 6 times faster from labwork1\_CPU() to labwork1\_OpenMP(). 

\end{document}

