\documentclass{article}
\usepackage[utf8]{inputenc}

\usepackage[latin1]{inputenc}
\usepackage[T1]{fontenc}
\usepackage{lmodern}
\usepackage[frenchb]{babel}
\usepackage{amsmath,mathrsfs,amssymb}


\title{LABWORK7}
\author{Damien ALvarez de Toledo}
\date{November 2019}

\begin{document}

\maketitle

\section{How we implemented the labwork :}\newline

Grey Scale Stretching processing was implemented with 4 different kernels: \newline
The first kernel is the Gray Scaling we usually use.\newline
After Grey Scaling is performed, we must find the minimum and maximum value pixels in the current block the kernel is being executed on. We thus implement a minimum/maximum search using parallel reduction with a shared memory inside our ReduceMin/ReduceMax kernel.\newline
Once the minimum and maximum are found, we execute the Gray Scale Stretching kernel that will calculate the new value for each pixel with the following formula :\newline

$g'=\dfrac{g - min}{max - min}\newline$



\end{document}

