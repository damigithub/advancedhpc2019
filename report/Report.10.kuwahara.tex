\documentclass{article}
\usepackage[utf8]{inputenc}

\title{LABWORK10}
\author{Damien ALVAREZ DE TOLEDO }
\date{November 2019}

\begin{document}

\maketitle

\section{Implementation of the Labwork}\newline

The Kuwahara labwork was implemented with one kernel : \textbf{kuwahara()}. In the kernel, we first create two shared memories, one for  saving the V of each pixel in our block, and the second to save the RGB data of each pixel of our block. We then find the mean value of all the V of each 4 sub-matrixes to calculate their standard deviation. After finding out which one has the minimum standard deviation, we operate the mean of RGBs of our pixels in that sub-matrix to then affect that value to our current pixel.\newline

\section{Results}\newline

Unfortunately this labwork is not functioning properly, the output image is an incomprehensible heap of colours.   


\end{document}

