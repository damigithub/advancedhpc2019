\documentclass{article}
\usepackage[utf8]{inputenc}

\usepackage[latin1]{inputenc}
\usepackage[T1]{fontenc}
\usepackage{lmodern}
\usepackage[frenchb]{babel}
\usepackage{amsmath,mathrsfs,amssymb}

\title{LABWORK8}
\author{Damien ALVAREZ DE TOLEDO }
\date{November 2019}

\begin{document}

\maketitle

\section{Implementation of the Labwork}\newline

Two new kernels were created for this labwork : \textbf{rgb2hsv} and \textbf{hsv2rgb}.\newline
For the \textbf{rgb2hsv} kernel, we converted the data of our rgb pixels to hsv data with the following formulas, after scaling our RGB pixels from [0...255] to [0...1] :\newline
\newline
\textbf{V = max(R,G,B)}
\newline
\newline 
With \textbf{min = min(R,G,B)} and \textbf{max = max(R,G,B)} we have :\newline

\textbf{delta = max - min}\newline
\newline

\textbf{$S=0$} if delta = 0,\newline
\newline

\textbf{$S=\dfrac{delta}{max}$} otherwise.\newline
\newline

And finally for H:
\newline
\newline
$H = 60 * \dfrac{G - B}{delta}mod6$ if max = R,
\newline
\newline
$H = 60 * (2 + \dfrac{B - R}{delta})$ if max = G,
\newline
\newline
$H = 60 * (4 + \dfrac{R - G}{delta})$ if max = B,
\newline
\newline
\newline

\newline
\newline

For the \textbf{hsv2rgb} kernel, we converted our hsv data back to rgb pixels with the following formulas :\newline
\newline
$d = \dfrac{H}{60}$\newline
\newline
$hi = (int)dmod6$\newline
\newline
$f = d - hi$\newline
\newline
$l = V * (1 - S) $\newline
\newline
$m = V * (1 - f*S)$\newline
\newline
$m = V * (1 - (1 - f)*S)$\newline
\newline
$(R,G,B) = (V,n,L)$ if (0<=H<60)\newline
$(R,G,B) = (m,V,L)$ if (60<=H<120)\newline
$(R,G,B) = (l,V,n)$ if (120<=H<180)\newline
$(R,G,B) = (l,m,V)$ if (180<=H<240)\newline
$(R,G,B) = (n,l,V)$ if (240<=H<300)\newline
$(R,G,B) = (V,l,m)$ if (300<=H<360)\newline
\newline 

And finally scaling from [0...1] to [0...255] has to be done onto the RGB triplet.\newline

\section{Comparison between input image and output image (after rgb2hsv and hsv2rgb kernels) :}\newline

If we take the eiffel tower image \textbf{eiffel.jpg} as an example, the image goes from a blue sky to a green sky, while the eiffel tower has some parts of magenta over it. Input and Output images are very similar.


\end{document}
